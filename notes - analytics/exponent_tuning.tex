\documentclass[superscriptaddress]{revtex4-1}
\usepackage{graphicx} % needed for figures
\graphicspath{ {Figures/} }
\usepackage{epstopdf}
\usepackage[caption=false]{subfig}

\usepackage{amsmath}
\usepackage{amsfonts}
\usepackage{amssymb}
\usepackage{bm}

\newcommand{\subscr}[2]{{#1}_{\textup{#2}}}
\newcommand{\supscr}[2]{{#1}^{\textup{#2}}}
\newcommand{\Ker}{\operatorname{Ker}}
%\newcommand{\Det}{\operatorname{Det}}
\newcommand{\Rank}{\operatorname{Rank}}
\newcommand{\Image}{\operatorname{Im}}
\newcommand{\real}{\mathbb{R}}
\newcommand{\complex}{\mathbb{C}}
%\DeclareMathOperator*{\argmin}{arg\,min}
\newcommand{\mc}{\mathcal}
\newcommand{\argmax}[2] {\mathrm{arg}\max_{#1}#2}
\newcommand{\argmin}[2] {\mathrm{arg}\min_{#1}#2}


\begin{document}
\title{Tuning Avalanche Exponent with Spontaneous Activity}
\date{\today}

\maketitle


\section{Mathematical Framework}
Consider a network $\mc G = (\mc V, \mc E)$, where $\mc V = {1, \dotsm, n}$, and $\mc E \subseteq \mc V \times \mc V$ are the sets of network vertices and directed edges. Let $A = [a_{ij}]$ be the weighted and directed adjacency matrix of $\mc G$. At each time point $t \in \mathbb{Z}_{\geq 0}$, we associate each node $i$ with a discrete non-negative random variable $X_i^t$.

The evolution of the dynamics in our system follow a hierarchical stochastic process. Starting at some non-random initial state $\bm{x}^0$, we define discrete non-negative random variable $X_{j}^t$ that represents the number of successful transmissions received by node $j$ at time $t$. Initially, we define
\begin{align*}
X_j^{t+1} \sim \sum_{i=1}^n B(X_i^{t},a_{ji}),
\end{align*}
which is the sum of binomial distributions conditioned on random variables that are the states at time $t$.



\section{Judicious Selection of Spontaneous Activity to Generate Avalanches}
We can enumerate all possible discrete states $\bm{x}_i$, as
\begin{align*}
\bm{x}_i \in 
\left( 
\begin{bmatrix}
0 \\ 0 \\ \vdots \\ 0
\end{bmatrix},
\begin{bmatrix}
0 \\ 0 \\ \vdots \\ 1
\end{bmatrix},
\dotsm,
\begin{bmatrix}
1 \\ 1 \\ \vdots \\ 1
\end{bmatrix},
\dotsm,
\begin{bmatrix}
d \\ d \\ \vdots \\ d
\end{bmatrix}
\right),
\end{align*}
and the probability of transitioning from any state $\bm{x}_m$ and $\bm{x}_p$ is given simply as the product of sums of binomial distributions. From this system, we can construct infinite-dimensional Markov system
\begin{align*}
\bm{p}(k) = \mathbb{T}\bm{p}(k-1),
\end{align*}
where $A \rightarrow \mathbb{T}$ is a map determined by the binomial evolution, and $\mathbb{T} = PDP^{-1}$. From $\bm{p}(0)$, we can decompose
\begin{align*}
\bm{p}(0) = \bm{e}_1 + c_2\bm{e}_2 + c_3\bm{e}_3 + \dotsm = P\bm{c},
\end{align*}
to yield
\begin{align*}
\bm{p}(t) = \bm{e}_1 + c_2\lambda_2^t\bm{e}_2 + c_3\lambda_3^t\bm{e}_3 + \dotsm = PD^tP^{-1}\bm{p}(0),
\end{align*}
where avalanche durations are given simply by the first entry of $\bm{p}(t)$. We find coefficients
\begin{align*}
\bm{c} = P^{-1}\bm{p}(0). 
\end{align*}

\subsection{Given}
By estimating $\mathbb{T}$, we obtain
\begin{align*}
\lambda(\mathbb{T})
\hspace{1in}
P = [\bm{e}_1, \dotsm],
\end{align*}
from which we know that $P(\bm{X}^t = \bm{0}) = 1 - c_2\lambda_2^te_{21} - c_3\lambda_3^te_{31} + \dotsm$. Of course, we can't pick any $\bm{c}$. We require
\begin{align*}
\bm{1}^TP\bm{c} = 1
\hspace{1in}
P\bm{c} \geq \bm{0}
\end{align*}

\subsection{Statement 1}
Hence, if we desire some $\bm{c}^*$, we seek
\begin{align*}
\argmin{\bm{c}}{(\bm{c}-\bm{c}^*)^T(\bm{c}-\bm{c}^*)},
\end{align*}
such that
\begin{align*}
\bm{1}^TP\bm{c}=1
\hspace{1in}
P\bm{c} \geq \bm{0}.
\end{align*}

\subsection{Statement 2}
A more direct statement is given $P(\bm{X}^t=\bm{0}) = 1 - |c_2||e_{21}||\lambda_2|^t - |c_3||e_{31}||\lambda_3|^t + \dotsm$, find 
\begin{align*}
\argmin{\bm{p}(0)}{\||P^{-1}\bm{p}(0)| - |\bm{c}^*|\|_2^2},
\end{align*}
such that
\begin{align*}
\sum \bm{p}(0) = 1
\hspace{1in}
\bm{p}(0) \geq 0.
\end{align*}

	
	
	
	
\end{document}
